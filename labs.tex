\documentclass[a4,11pt]{article}
\usepackage{sectionbox}
\usepackage{hyperref}
\usepackage{graphicx}

%\usepackage{upquote}
\usepackage{listings}

\setlength{\topmargin}{-2cm}
\setlength{\textwidth}{16.5cm}
\setlength{\textheight}{24cm}
\setlength{\evensidemargin}{0cm}
\setlength{\oddsidemargin}{0cm}

\lstset{basicstyle=\ttfamily, keywordstyle=\color{blue}\ttfamily,
  stringstyle=\color{red}\ttfamily,
  commentstyle=\color{magenta}\ttfamily,
  morecomment=[l][\color{magenta}]{\#}}


\newcommand{\postit}[1]{%
  \noindent
  \fcolorbox{red}{yellow}{%
    \begin{minipage}{5cm}
      #1
    \end{minipage}
   }
}


\title{\textsc{Big Data Analytics (SOEN 498/691)} \\ Laboratory sessions}

\author{Tristan Glatard\\Department of Computer Science and Software Engineering\\Concordia University, Montreal\\\url{mailto:tristan.glatard@concordia.ca}}


\begin{document}

\maketitle

\newpage

\tableofcontents

\newpage

\part{Prerequisites}

\section{Java}

\postit{Java installation}

Make sure that you are able to create and run a simple Java program
before you start. The examples described in this tutorial can be done
using a simple text editor such as \texttt{vim} or \texttt{emacs} and
Linux command-line tools. Here is a simple Java program to help you
start:\\
\begin{sectionbox}{}
  \begin{lstlisting}[language=Java]
    public class Test{
      public static void main(String[] args){
        System.out.println(''Hello, World!'');
      }
    }
\end{lstlisting}
\end{sectionbox}
It is compiled as follows:\\
\begin{sectionbox}{}
\begin{verbatim}
  % javac Test.java
\end{verbatim}
\end{sectionbox}
The compilation produces a \texttt{Test.class} file that contains the compiled class.
The program is executed as follows:\\
\begin{sectionbox}{}
\begin{verbatim}
  % java Test
\end{verbatim}
\end{sectionbox}
\texttt{Test.class} must be located in the directory where \texttt{java} is
executed, or in a directory listed in the \texttt{CLASSPATH}
environment variable.

However, for larger applications such as your project, it is
recommended to use Integrated Development Environments (IDEs) tailored
for Java, e.g., Eclipse or Netbeans.

\section{Linux}

Here is a list of Linux commands that are required to complete this
tutorial. Make sure that you understand them before you proceed.
\begin{itemize}
  \item \texttt{man}:
  \item \texttt{ls}:
  \item \ldots
\end{itemize}

Here is also a list of environment variables that will have to be
used:
\begin{itemize}
  \item \texttt{PATH}:
  \item \texttt{CLASSPATH}: 
\end{itemize}

\part{Getting started with Hadoop}

\section{Installation (standalone mode)}

See \url{http://hadoop.apache.org/docs/r2.7.3/hadoop-project-dist/hadoop-common/SingleCluster.html#Standalone_Operation}.

\section{First example}

Adapt from \url{http://hadoop.apache.org/docs/r2.7.3/hadoop-mapreduce-client/hadoop-mapreduce-client-core/MapReduceTutorial.html#Example:_WordCount_v1.0}.

\section{Hadoop Streaming}

\section{Installation (pseudo-distributed mode)}

\section{HDFS}

\section{YARN}

\section{Second example}


  \begin{sectionbox}{}
   \begin{verbatim}
rm * -Rf
   \end{verbatim}
  \end{sectionbox}


\end{document}


