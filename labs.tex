\documentclass[11pt]{article}

% Packages
\usepackage[colorlinks]{hyperref}
\usepackage{listings}
\usepackage{graphicx}
\usepackage[dvipsnames]{xcolor}

% Colors
\definecolor{mygray}{RGB}{235,235,235}

% Margins
\setlength{\topmargin}{-2cm}
\setlength{\textwidth}{16.5cm}
\setlength{\textheight}{24cm}
\setlength{\evensidemargin}{0cm}
\setlength{\oddsidemargin}{0cm}

% Fix link colors
\hypersetup{
    colorlinks = true,
    linkcolor=red,
    citecolor=red,
    urlcolor=blue,
    linktocpage % so that page numbers are clickable in toc
}


% Code listings
\lstset{
  basicstyle=\ttfamily,
  keywordstyle=\color{blue}\ttfamily,
  stringstyle=\color{red}\ttfamily,
  commentstyle=\color{magenta}\ttfamily,
  morecomment=[l][\color{magenta}]{\#}
}

\lstnewenvironment{cli}
                  {\footnotesize
                    \lstset{columns=fullflexible,
                      language=bash,
                      backgroundcolor=\color{mygray}
                  }}
{}

\lstnewenvironment{xml}
                  {\footnotesize
                    \lstset{columns=fullflexible,
                      language=XML,
                      backgroundcolor=\color{Salmon},
                      morekeywords={property,name,value,description,configuration}
                  }}
{}

\newcommand{\bashcode}[1]{
  \begin{footnotesize}
  \par
  \hfill \colorbox{SkyBlue}
         {
           \href{https://github.com/glatard/big-data-analytics-labs/raw/master/#1}
                {(\underline{Link to file})
         }} \hfill
         \lstset{language=bash,
           columns=fullflexible,
           backgroundcolor=\color{SkyBlue}}
  \vspace*{-0.3cm}
  \lstinputlisting{#1}
  \end{footnotesize}
}

\newcommand{\javacode}[1]{
  \begin{footnotesize}
  \par
  \hfill \colorbox{YellowGreen}
         {\href{https://github.com/glatard/big-data-analytics-labs/raw/master/#1}
           {(\underline{Link to file})
         }} \hfill
         \lstset{language=java,
           columns=fullflexible,
           backgroundcolor=\color{YellowGreen}}
  \vspace*{-0.3cm}
  \lstinputlisting{#1}
  \end{footnotesize}
}
               
% Notes and TODOs              
\newcommand{\postit}[1]{%
%  \noindent
%  \fcolorbox{red}{yellow}{%
%    \begin{minipage}{5cm}
%      #1
%    \end{minipage}
%   }
}

\title{\textsc{Big Data Analytics (SOEN 498/691)} \\ Laboratory sessions}

\author{Tristan Glatard\\Department of Computer Science and Software Engineering\\Concordia University, Montreal\\\href{mailto:tristan.glatard@concordia.ca}{tristan.glatard@concordia.ca}}

\begin{document}

\maketitle

\newpage

\tableofcontents

\newpage

\part{Prerequisites}

\section{Notations}

This is a command you are supposed to type in a Linux terminal (don't type the \$ sign):
\begin{cli}
  $ echo Welcome
\end{cli}
This command is split in two lines:
\begin{cli}
  $ echo Welcome to this tutorial,\
    I hope you will enjoy it! 
\end{cli}
This is a piece of Java code:
\javacode{code/line.java}
This is a piece of bash script:
\bashcode{code/test.sh}
And this is a piece of XML file:
\begin{xml}
  <one>
    <two>Hello</two>
  </one>
\end{xml}

\section{Java}

Java is the primary language used in Hadoop. Make sure that Java
version 7 or higher is installed on your system. If it is not, follow
the instructions from
\href{http://www.oracle.com/technetwork/java/javase/downloads/jdk8-downloads-2133151.html}{the
  Oracle website} and check that the Java Virtual Machine (JVM) works correctly:
\begin{cli}
$ java -version
\end{cli}
Make sure that you are able to create and run a simple Java program
such as:
\javacode{code/Test.java}
This program is compiled as follows:
\begin{cli}
$ javac Test.java
\end{cli}
The compilation produces a \texttt{Test.class} file that contains the compiled class.
The program can be executed as follows:
\begin{cli}
$ java Test
\end{cli}
\texttt{Test.class} must be located in the directory where \texttt{java} is
executed, or in a directory listed in the \texttt{CLASSPATH}
environment variable.

This tutorial can be done with Java command-line tools
(\texttt{javac} and \texttt{java}) a text editor such as \texttt{vim}
or \texttt{emacs}. For larger applications, for instance your course
project, it is recommended to use an Integrated Development
Environment (IDE) tailored for Java, e.g.,
\href{http://eclipse.org}{Eclipse} or
\href{http://netbeans.org}{Netbeans}.

\section{Linux}
This tutorial is based on Linux and it uses the
\href{https://www.gnu.org/software/bash/}{bash} shell. You don't have
to install bash as it is already present on your workstation.

\subsection{Useful commands}
\begin{itemize}
\item \texttt{cat}: prints the content of a file on the standard output.
\item \texttt{chmod}: changes the access permissions of a file or directory.
\item \texttt{echo}: prints a message on the standard output.
\item \texttt{export}: exports environment variables to the
  environment of subsequently executed commands.
\item \texttt{ls}: lists directory content.
\item \texttt{man}: gets help on any command.
\item \texttt{mkdir}: creates a directory.
\item \texttt{seq 1 1000}: prints the integers from 1 to 1000. 
\item \texttt{tar}: handles 
  \texttt{tar}, \texttt{tgz} and many other types of archives.
\item \texttt{wget}: downloads the content of a URL.
\end{itemize}

\subsection{Standard streams}

A Linux process is connected to three particular files called \emph{streams}:
\begin{itemize}
\item The standard \emph{output} is used by the process to write normal information.
\item The standard \emph{error} is used by the process to write error information.
\item The standard \emph{input} is used to send information to the process.
\end{itemize}
We will use streams extensively in the tutorial.

\subsection{Useful operators}
\begin{itemize}
  \item \texttt{>} (redirection operator): redirects the standard
    output of its left argument to a file. For instance, \texttt{echo
      Hello > a.txt} writes ``Hello'' in file \texttt{a.txt} (if
    \texttt{a.txt} already exists, it is overwritten).
    \item \texttt{>>}: same as \texttt{>} but the standard output of
      the left operand is \emph{appended} to the file in the right operand.
  \item \texttt{|} (pipe operator): redirects the standard output of
    its left argument to the standard input of its right argument. For
    instance, \texttt{cat a.txt | sort} prints the sorted content of a
    file.
\end{itemize}
Here is also a list of useful environment variables:
\begin{itemize}
  \item \texttt{PATH}: tells the system which directories to search for executables.
  \item \texttt{CLASSPATH}: tells the JVM where classes must be searched.
\end{itemize}

\newpage

\part{Getting started with Hadoop}

This part of the tutorial will guide you through practical steps to
install Hadoop and write simple MapReduce programs. For a theoretical
presentation of Hadoop and MapReduce, refer to Chapter 2
of~\cite{mmds} (available
\href{http://infolab.stanford.edu/~ullman/mmds/ch2.pdf}{here}) or to
Chapter 2 of~\cite{lin} (available
\href{http://lintool.github.io/MapReduceAlgorithms/ed1n/MapReduce-algorithms.pdf}{here})
.  The content presented hereafter is adapted from the following
sources:
\begin{itemize}
\item \href{http://hadoop.apache.org/docs/r2.7.3/index.html}{Apache Hadoop documentation}
\item \href{http://hadoop.apache.org/docs/r2.7.3/hadoop-mapreduce-client/hadoop-mapreduce-client-core/MapReduceTutorial.html}{MapReduce tutorial}
\item \href{http://hadoopbook.com}{Tom White, ``Hadoop: The Definitive Guide''. Chapter 2 (MapReduce).}
\end{itemize}

For now we will run Hadoop in \emph{standalone mode}, i.e., in a
single JVM. This mode is useful for debugging programs and testing
things out.

\section{Installation}

Download Hadoop release 2.7.3 from one of the
\href{http://www.apache.org/dyn/closer.cgi/hadoop/common/}{Apache
  Download Mirrors} and unpack the distribution:
\begin{cli}
  $ wget http://apache.mirror.rafal.ca/hadoop/common/hadoop-2.7.3/hadoop-2.7.3.tar.gz
  $ tar xf hadoop-2.7.3.tar.gz
  $ export HADOOP=$PWD/hadoop-2.7.3
\end{cli}
Edit \texttt{etc/hadoop/hadoop-env.sh} and define the path to your Java installation:
\begin{cli}
  export JAVA_HOME=/path/to/your/java/installation/
\end{cli}
Add the Hadoop executables to your PATH:
\begin{cli}
  $ export PATH=$PATH:${HADOOP}/bin:${HADOOP}/sbin
\end{cli}
Make sure Hadoop is installed properly:
\begin{cli}
  $ hadoop version
\end{cli}
Run a first example:
\begin{cli}
  $ mkdir input
  # creates a test file containing integers from 1 to 1000
  $ for i in `seq 1 1000`; do echo $i >> input/file.txt ; done 
  $ hadoop jar ${HADOOP}/share/hadoop/mapreduce/hadoop-mapreduce-examples-2.7.3.jar grep input output 1
  $ cat output/*
\end{cli}
This program prints the lines of the input file that contain the
string ``1''.

\section{First MapReduce program}
\label{sec:word-count}

As a first MapReduce program, write the classical WordCount example:
\javacode{WordCount.java}
The code above is commented so that you should be able to
understand what it does. Refer to the
\href{http://hadoop.apache.org/docs/r2.7.3/api/index.html}{Apache
  Hadoop API reference} if anything is unclear.

We will now compile and run this program. Add Hadoop's jars to the
CLASSPATH:
\begin{cli}
  export CLASSPATH=$CLASSPATH:${HADOOP}/share/hadoop/common/hadoop-common-2.7.3.jar:\
             ${HADOOP}/share/hadoop/mapreduce/hadoop-mapreduce-client-core-2.7.3.jar:.
\end{cli}
Compile the program:
\begin{cli}
$ javac WordCount.java
\end{cli}
Create a jar containing all the classes in WordCount:
\begin{cli}
  $ jar cvf wordcount.jar WordCount*.class
\end{cli}
Create a test file and run the program:
\begin{cli}
  $ echo one two three two three three > test.txt
  $ hadoop jar wordcount.jar WordCount test.txt output
\end{cli}

\section{Hadoop Streaming}
\label{sec:hadoop-streaming}

While the Hadoop framework is developed in Java, it can also execute
programs written in any language using the Hadoop Streaming tool. In
this section, we are going to implement the WordCount example in
bash. You may also implement it in any other language (Python,
Ruby, C, etc) if you wish.

Hadoop Streaming works as follows:
\begin{enumerate}
\item The key/value pairs are passed to the mapper executable through its standard input.
\item The mapper executable writes output key/value pairs to its standard output.
\item The sorted key/value pairs are passed to the reduce executable
  through its standard input.
\end{enumerate}

Hence, mappers and reducers can be tested as follows for any MapReduce
program executed with Hadoop Streaming:
\begin{cli}
  $ cat input.txt | mapper.sh | sort | reduce.sh
\end{cli}
(Replace \texttt{input.txt} with the name of your input file,
\texttt{mapper.sh} with the bash script containing the mapper code and
\texttt{reduce.sh} with the bash script containing the reduce code).
The mapper code is straightforward:
\bashcode{code/wordcount-mapper.sh} The reducer code is a bit more
complex: \bashcode{code/wordcount-reducer.sh} Run the program with
Hadoop Streaming as follows:
\begin{cli}
  hadoop jar ${HADOOP}/share/hadoop/tools/lib/hadoop-streaming-2.7.3.jar\
  -input test.txt\
  -output output\
  -mapper ./wordcount-mapper.sh\
  -reducer ./wordcount-reducer.sh\
  -file wordcount-reducer.sh -file wordcount-mapper.sh 
\end{cli}
Create a larger text file and run the program again. The number of
mappers and reducers can be changed with the following options:
\texttt{mapreduce.job.reduces} and \texttt{mapreduce.job.maps}. For
instance, to use two mappers and two reducers:
\begin{cli}
  hadoop jar ${HADOOP}/share/hadoop/tools/lib/hadoop-streaming-2.7.3.jar \
  -D mapreduce.job.reduces=2 \
  -D mapreduce.job.map=2 \
  -input test.txt \
  -output output \
  -mapper ./mapper.sh \
  -reducer ./reducer.sh \
  -file reducer.sh -file mapper.sh 
\end{cli}
The framework still uses only a single mapper and a single reducer
because it is configured in standalone mode, i.e., as a single Java
process. In the next part of the tutorial, we will configure the
pseudo-distributed mode where Hadoop runs in several processes.
%Funny enough, the number of mappers is ignored (because job tracker
%is set to local) while the number of reducers is not.

\postit{Add combiner}

\part{HDFS and YARN}


In standalone mode, Hadoop was using the local file system to store
files and all the jobs were executed in the same JVM. In
pseudo-distributed mode, it uses the Hadoop Distributed File System
(HDFS) and YARN (Yet Another Resource Negotiator).

The content presented hereafter is adapted from the following
sources:
\begin{itemize}
\item \href{http://hadoop.apache.org/docs/r2.7.3/index.html}{Apache Hadoop documentation}
\item \href{http://hadoop.apache.org/docs/r2.7.3/hadoop-mapreduce-client/hadoop-mapreduce-client-core/MapReduceTutorial.html}{MapReduce tutorial}
\item \href{http://hadoopbook.com}{Tom White, ``Hadoop: The Definitive Guide''. Chapter 2 (MapReduce), Chapter 3 (HDFS).}
\end{itemize}

\section{ssh setup}

Make sure that you can connect to your own workstation using ssh
\emph{with no password}:
\begin{cli}
  $ ssh localhost
\end{cli}
If the ssh command prompts for a password, create a password-less ssh key and authorize it in your account:
\begin{cli}
  $ mkdir -p $HOME/.ssh
  $ chmod 700 $HOME/.ssh
  $ ssh-keygen
  # press 'Enter' at every question
  $ cat $HOME/.ssh/id_rsa.pub >> $HOME/.ssh/authorized_keys
  $ chmod 600 $HOME/authorized_keys
\end{cli}

\section{HDFS}

\subsection{Configuration}

Configure the HDFS by adding the following to \texttt{etc/hadoop/core-site.xml}:
\begin{xml}
<configuration>
    <property>
        <name>fs.defaultFS</name>
        <value>hdfs://localhost:9000</value>
    </property>
    <property>
        <name>hadoop.tmp.dir</name>
        <value>/some/path/on/your/workstation</value>
    <description>A base for other temporary directories.</description>
  </property>
</configuration>
\end{xml}
and to \texttt{etc/hadoop/hdfs-site.xml}:
\begin{xml}
<configuration>
    <property>
        <name>dfs.replication</name>
        <value>1</value>
    </property>
</configuration>
\end{xml}
Format the file system:
\begin{cli}
$  hdfs namenode -format
\end{cli}
And start the HDFS daemons:
\begin{cli}
$ start-dfs.sh
\end{cli}

\subsection{Command-line usage}

We will now review the main HDFS commands. Directory creation:
\begin{cli}
  $ hdfs dfs -mkdir /user
  $ hdfs dfs -mkdir /user/<username>
\end{cli}
Directory listing:
\begin{cli}
  $ hdfs dfs -ls /user
\end{cli}
File copy to HDFS:
\begin{cli}
  $ echo This is a test > test.txt
  $ hdfs dfs -put test.txt /user/<username>/test.txt
\end{cli}
File copy from HDFS:
\begin{cli}
  $ hdfs dfs -get /user/<username>/test.txt
\end{cli}
File content display:
\begin{cli}
  $ hdfs dfs -cat /user/<username>/test.txt
\end{cli}
File removal:
\begin{cli}
  $ hdfs dfs -rm /user/<username>/test.txt
\end{cli}

\subsection{Web interface}

HDFS can be browsed through a web interface which is by default
available at \href{http://localhost:50070}{http://localhost:50070}.

\subsection{Java API}

The \texttt{FileSystem} API is the safest way to access files on HDFS
from a Java program. A file in a Hadoop file system is represented by a
Hadoop \texttt{Path} object. You can think of a Path as a Hadoop
file system URI, such as
\texttt{hdfs://localhost/user/tom/quangle.txt}.

The following program reads a file on HDFS:
\javacode{code/HdfsCat.java}
After having compiled this file, create a jar and run it with Hadoop:
\begin{cli}
  $ hadoop jar <path to jar> HdfsCat <HDFS URI of file to cat>
\end{cli}
A similar program can be written to write files to HDFS:
\javacode{code/HdfsPut.java}
Compile this class and make sure you can use it to write files to HDFS.
% distcp?

\section{YARN}

YARN (Yet Another Resource Negotiator) is the resource management
system used in pseudo-distributed mode.

\subsection{Configuration}

Edit \texttt{etc/hadoop/mapred-site.xml} to add the following configuration properties:
\begin{xml}
  <configuration>
    <property>
      <name>mapreduce.framework.name</name>
      <value>yarn</value>
    </property>
  </configuration>
\end{xml}
Edit \texttt{etc/hadoop/yarn-site.xml} to add the following configuration properties:
\begin{xml}
  <configuration>
    <property>
      <name>yarn.nodemanager.aux-services</name>
      <value>mapreduce_shuffle</value>
    </property>
    <property>
      <name>yarn.nodemanager.aux-services.mapreduce.shuffle.class</name>
      <value>org.apache.hadoop.mapred.ShuffleHandler</value>
    </property>
    <property>
      <name>yarn.nodemanager.vmem-check-enabled</name>
      <value>false</value>
      <description>Whether virtual memory limits will be enforced for containers.</description>
    </property>
  </configuration>
\end{xml}
Start the YARN daemon:
\begin{cli}
  $ start-yarn.sh
\end{cli}
The MapReduce job history daemon is another useful service to start:
\begin{cli}
  $ mr-jobhistory-daemon.sh start historyserver
\end{cli}

Re-run the Hadoop Streaming example of
Section~\ref{sec:hadoop-streaming} using multiple mappers and
reducers.

\postit{benchmark with multiple mappers}

\subsection{Web interfaces}

Browse the YARN monitoring interface (\url{http://localhost:8088})
and the job history server (\url{http://localhost:19888}).

\subsection{Improved WordCount}

Here is a more complete WordCount example that adds the following
features to the initial example presented in Section~\ref{sec:word-count}:
\begin{itemize}
\item It uses the distributed cache to distribute a ``skip'' file to all the mappers. The skip file contains patterns to be removed from words.
\item The Mapper uses a \texttt{setup} method.
\item It uses Counters to track statistics related to the execution.
\item It uses Configuration to read configuration parameters.  
\end{itemize}
The code below has been commented so that you should be able to
understand what it does. Refer to the
\href{http://hadoop.apache.org/docs/r2.7.3/api/index.html}{Apache
  Hadoop API reference} if anything is unclear.

\javacode{code/WordCount2.java}
Compile this example and check that it works as expected:
\begin{itemize}
\item Create a
  text file with uppercase and lowercase characters and special
  characters such as ',', '.' and '!'.
\item Create a skip file containing the
  special characters.
\item Make sure that the program counts words as if there was no
  special character.
\item Make sure that the INPUT\_WORDS counter is properly
  reported in the standard output at the end of the execution.
\item Use option -Dwordcount.case.sensitive=false to perform case-insensitive matching.
\end{itemize}

Troubleshooting:
\begin{itemize}
\item Jar
\texttt{${HADOOP}/share/hadoop/hdfs/lib/commons-cli-1.2.jar} must be added to the \texttt{CLASSPATH} variable to compile the
code.
\item Special characters such as '.' must be
  escaped as in '\.' in the skip file.
  \end{itemize}

\bibliographystyle{plain} \bibliography{biblio}

\end{document}


