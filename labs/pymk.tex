\documentclass[11pt]{article}

% Packages
\usepackage[colorlinks]{hyperref}
\usepackage{listings}
\usepackage{graphicx}
\usepackage[dvipsnames]{xcolor}
\usepackage{amsmath}

% Colors
\definecolor{mygray}{RGB}{235,235,235}

% Margins
\setlength{\topmargin}{-2cm}
\setlength{\textwidth}{16.5cm}
\setlength{\textheight}{24cm}
\setlength{\evensidemargin}{0cm}
\setlength{\oddsidemargin}{0cm}

% Fix link colors
\hypersetup{
    colorlinks = true,
    linkcolor=red,
    citecolor=red,
    urlcolor=blue,
    linktocpage % so that page numbers are clickable in toc
}


% Code listings
\lstset{
  basicstyle=\ttfamily,
  keywordstyle=\color{blue}\ttfamily,
  stringstyle=\color{red}\ttfamily,
  commentstyle=\color{magenta}\ttfamily,
  morecomment=[l][\color{magenta}]{\#}
}


\lstnewenvironment{cli}
                  {\footnotesize
                    \lstset{columns=fullflexible,
                      language=bash,
                      backgroundcolor=\color{mygray}
                  }}
{}

\lstnewenvironment{xml}
                  {\footnotesize
                    \lstset{columns=fullflexible,
                      language=XML,
                      backgroundcolor=\color{Salmon},
                      morekeywords={property,name,value,description,configuration}
                  }}
{}

\newcommand{\bashcode}[1]{
  \begin{footnotesize}
  \par
  \hfill \colorbox{SkyBlue}
         {
           \href{https://github.com/glatard/big-data-analytics-labs/raw/master/labs/#1}
                {(\underline{Link to file})
         }} \hfill
         \lstset{language=bash,
           columns=fullflexible,
           backgroundcolor=\color{SkyBlue}}
  \vspace*{-0.3cm}
  \lstinputlisting{#1}
  \end{footnotesize}
}

\newcommand{\javacode}[1]{
  \begin{footnotesize}
  \par
  \hfill \colorbox{YellowGreen}
         {\href{https://github.com/glatard/big-data-analytics-labs/raw/master/labs/#1}
           {(\underline{Link to file})
         }} \hfill
         \lstset{language=java,
           columns=fullflexible,
           backgroundcolor=\color{YellowGreen}}
  \vspace*{-0.3cm}
%  \lstinputlisting{#1}
  \end{footnotesize}
}

\newcommand{\textfile}[1]{
  \begin{footnotesize}
  \par
  \hfill \colorbox{SkyBlue}
         {\href{https://github.com/glatard/big-data-analytics-labs/raw/master/labs/#1}
           {(\underline{Link to file})
         }} \hfill
         \lstset{columns=fullflexible,
           backgroundcolor=\color{SkyBlue}}
  \vspace*{-0.3cm}
  \lstinputlisting{#1}
  \end{footnotesize}
}


% Notes and TODOs              
\newcommand{\postit}[1]{%
%  \noindent
%  \fcolorbox{red}{yellow}{%
%    \begin{minipage}{5cm}
%      #1
%    \end{minipage}
%   }
}

\title{\textsc{Big Data Analytics (SOEN 498/691)} \\ Laboratory sessions}

\author{Tristan Glatard\\Department of Computer Science and Software Engineering\\Concordia University, Montreal\\\href{mailto:tristan.glatard@concordia.ca}{tristan.glatard@concordia.ca}}

\begin{document}

\maketitle

\newpage

\tableofcontents

\newpage

\section{Introduction}

Now that you are able to run a simple MapReduce example using Hadoop,
we will focus on more complex algorithms. We will look at two popular
applications used in social networks: ``People You May Know''
(Part~\ref{part:pymk} below) and ``Who To Follow''
(Part~\ref{part:wtf}).  ``People You May Know'' is not an
assignment. Links to solutions implemented in Java are provided but it
is strongly recommended that you try to implement your own code before
looking at the solutions. ``Who To Follow'' is an assignment due for
the date indicated on the course outline. It is strongly recommended
to complete ``People You May Know'' before doing ``Who To Follow'' as
the solutions to these problems have a lot in common.

\part{People You May Know}
\label{part:pymk}

\section{Goal}

We will implement a basic ``People You May Know'' algorithm using
MapReduce. We assume an un-directed social network of \emph{n} users
with a symmetrical ``friend'' relation among users, that is, if user
\texttt{A} is a friend of user \texttt{B} then user \texttt{B} is also
a friend of user \texttt{A}. Facebook is an example of such a
network. Twitter is a counter example because user \texttt{A} may
follow user \texttt{B} whereas user \texttt{B} does not follow user
\texttt{A}. The algorithm will recommended to user \texttt{X} a list
of friends \texttt{Y$_i$} based on the number of friends that
\texttt{X} and \texttt{Y$_i$} have in common. The input and output of the
algorithm will be as follows:
\begin{itemize}
\item Input: a file containing \emph{n} lines with space-separated integers:
  \newline \texttt{X F$_1$ F$_2$ ...}\newline where the
  \texttt{F$_i$} are the ids of the friends of user
  \texttt{X}. For instance: \textfile{pymk/data-small.txt}
\item Output: a file containing \emph{n} lines in the following
  format:\newline \texttt{X R$_1$(n$_1$) R$_2$(n$_2$) R$_3$(n$_3$)
    ...}\newline where \texttt{R$_i$} are the ids of the friends
  recommended to user \texttt{X} and $n_i$ is the number of friends in
  common between \texttt{X} and \texttt{R$_i$}. Recommended friends
  \texttt{R$_i$} \textbf{must be} ordered by decreasing values of
  \texttt{n$_i$} and they \textbf{must not} be friends of \texttt{X}. For
  instance: \textfile{pymk/output-pymk-1/part-r-00000}
\end{itemize}

The following Sections will guide you through a possible
implementation. Feel free to follow your own
thread if you have any idea how to implement it: you may find a
better solution than the one proposed!

\section{Basic idea}

The basic idea of our initial implementation is the following:
\begin{itemize}
\item Map step:
  \begin{itemize}
  \item Receive (\_,v): \texttt{\_, X F$_1$ F$_2$ ... F$_k$}
  \item Emit (k,v): \texttt{F$_i$},\texttt{F$_j$} where $i \in [1,k]$, $j \in [1,k]$ and $i \neq j$.
  \end{itemize}
  In other words, \texttt{(a,b)} \emph{and} \texttt{(b,a)} are emitted every time a friend common to user \texttt{a} and user \texttt{b} is found. The following key-value pairs are emitted in response to the input example above (order may of course vary):
  \textfile{pymk/emitted.txt}
\item Reduce step:
  \begin{itemize}
  \item Receive (k, [v]): \texttt{X, [ F$_1$, F$_2$
    ... ]} \newline In this list the \texttt{F$_i$} are not
    unique: instead, \texttt{F$_i$} appears exactly \texttt{x} times
    when user \texttt{X} and user \texttt{F$_i$} have \texttt{x}
    friends in common. For instance, the following key-value pairs are received in our example:
    \textfile{pymk/received.txt}
  \item Emit (k, v): \texttt{X, "F$_1$(n$_1$) F$_2$(n$_2$) ..."}\newline
    where \texttt{F$_i$} appeared exactly \texttt{n$_i$} times in the received values. In our example: \textfile{pymk/output-pymk/part-r-00000}
  \end{itemize}
\end{itemize}

\section{First implementation}

Implement the algorithm described in the previous Section. Make sure
that it produces the correct result on the example above. Then, test
your implementation on larger networks by generating input files using
a program such as
\href{https://github.com/glatard/big-data-analytics-course/raw/master/labs/pymk/generate.py}{\texttt{generate.py}}.

Here is a possible solution:
\javacode{pymk/Pymk.java}

\section{Filtering existing friends}

There is a problem with the previous implementation as it may
recommend to user \texttt{X} people that are already friend with
\texttt{X}. For instance, in the example above, user \texttt{3} is
recommended to user \texttt{1} although \texttt{1} and \texttt{3} are
already friends. We need to somehow emit key-value pairs containing
the friends of user \texttt{i}, to address this issue.  Here is a
possible solution:
\begin{itemize}
\item Map step:
  \begin{itemize}
  \item Receive (\_,v): \emph{no change}.
  \item Emit (k,v):
    \begin{itemize}
    \item \texttt{F$_i$},\texttt{F$_j$} where $i \in [1,k]$, $j \in [1,k]$ and $i \neq j$.
    \item \texttt{X}, \color{red}\textbf{-}\color{black}\texttt{F$_i$} for all $i \in [1,k]$ (note the \texttt{-} sign). 
    \end{itemize}
    For instance, in the example above: \textfile{pymk/emitted-1.txt}
  \end{itemize}
\item Reduce step:
  \begin{itemize}
  \item Receive (k, [v]): \texttt{X, [ F$_1$, F$_2$, ... ]} \newline
    
    If \texttt{F$_i$} is negative then it is a friend of user
    \texttt{X} and all occurrences of -\texttt{F$_i$} in the values
    must be removed. Otherwise, \texttt{F$_i$} is a user who has
    friends in common with \texttt{X} and its number of occurrences
    must be counted as before. In our example:
    \textfile{pymk/received-1.txt}
  \item Emit (k, v): \texttt{X, "F$_1$(n$_1$) F$_2$(n$_2$)
    ..."}\newline where \texttt{F$_i$} appeared exactly \texttt{$n_i$}
    times in the received values after the friends of \texttt{X}
    have been removed. In our example:
    \textfile{pymk/output-pymk-1/part-r-00000}
  \end{itemize}
\end{itemize}

Implement this solution and verify that it works correctly.

Here is a possible solution:
\javacode{pymk/Pymk1.java}

\newpage

\part{Who To Follow}
\label{part:wtf}

\section{Goal}
\label{sec:goal-wtf}
We now assume a directed social network of \emph{n} users with an
asymmetrical ``follows'' relation among users, that is, user B doesn't
necessarily follows user \texttt{A} even though user \texttt{A}
follows user \texttt{B}. Twitter is an example of such a network. The
goal is to write an algorithm that will recommend to user \texttt{X} a
list of people to follow \texttt{F$_i$} based on the number of
followers that \texttt{X} and \texttt{F$_i$} have in common. The input
and output of the algorithm will be as follows:
\begin{itemize}
\item Input: a file containing \emph{n} lines with space-separated integers:
  \newline \texttt{X F$_1$ F$_2$ ...}\newline where
  \texttt{F$_i$} are followed by user
  \texttt{X}. For instance: \textfile{wtf/data-small.txt}
\item Output: A file containing \emph{n} lines in the following
  format: \newline \texttt{X R$_1$(n$_1$) R$_2$(n$_2$) R$_3$(n$_3$) ...}\newline where
  \texttt{R$_i$} are the ids of the people recommended to user
  \texttt{X} and $n_i$ is the number of followed people in common
  between \texttt{X} and \texttt{R$_i$}. Recommended people
  \texttt{R$_i$} \textbf{must not} be followed by \texttt{X} and \textbf{must} be
  ordered by decreasing values of \texttt{n$_i$}. On the previous example: \textfile{wtf/output-wtf-counter/part-r-00000}
\end{itemize}

\section{Basic idea}
\label{sec:idea-wtf}
Similarly to what we did for ``People You May Know'', you may want to
write a mapper that emits (\texttt{a},\texttt{b}) and
(\texttt{b},\texttt{a}) every time user \texttt{a} and user \texttt{b}
follow someone in common. However, because of the fact that the ``follows'' relation is
asymmetrical, this requires two MapReduce jobs:
\begin{enumerate}
  \item Indexing: for each user \texttt{F$_i$} followed by user
    \texttt{X}, the mapper emits \texttt{F$_i$} as the key and \texttt{X}
    as the value. The reducer is the identity. It produces inverted
    lists of followers: \newline \texttt{X}, [ \texttt{Y$_1$}, \texttt{Y$_2$},
    \texttt{...} , \texttt{Y$_k$} ]\newline where the Y$_i$ all follow user
    \texttt{X}.
  \item Similarity: for each inverted list \texttt{X}, [
    \texttt{Y$_1$}, \texttt{Y$_2$}, \texttt{...}, \texttt{Y$_k$} ] the
    mapper emits all pairs (\texttt{Y$_i$},\texttt{Y$_j$}) and
    (\texttt{Y$_j$},\texttt{Y$_i$}) where $i \in [1,k]$, $j \in [1,k]$
    and $i \neq j$. As in ``People You May Know'', the reducer
    receives a list \texttt{X}, [ \texttt{F$_1$, F$_2$, ...} ] where
    F$_i$ appears exactly \texttt{x} times if \texttt{X} and
    \texttt{F$_i$} follow \texttt{x} people in common. It counts the
    occurrences of \texttt{F$_i$} whenever \texttt{F$_i$} is not
    followed by \texttt{X} and sorts the resulting recommendations by
    number of common followed people.
\end{enumerate}

To filter the list of suggested people from followed people, you can
use the same kind of trick as for ``People You May Know'', using
negative integers.

\section{Assignment}

Write a MapReduce program that implements a basic ``Who To Follow'' system
fulfilling the requirements described in Section~\ref{sec:goal-wtf}. You may:
\begin{itemize}
\item Follow the
  idea in~\ref{sec:idea-wtf} or your own idea.
\item Use Java or any other programming language (using Hadoop
  Streaming).
\end{itemize}
You must:
\begin{itemize}
\item Make sure that your program will work on Linux.
\item Include a README file explaining how to compile (if relevant)
  your program, how to run it and where to find the results. Mention all the
  requirements, e.g., the version of Java required.
\item Use Git to track your developments. Your commit history will be
  checked and frequent commits showing incremental development will be
  favored over monolithic ones.
\end{itemize}

Your program will be tested on the example data of
Section~\ref{sec:goal-wtf} as well as on undisclosed evaluation
data generated with \href{https://github.com/glatard/big-data-analytics-course/raw/master/labs/wtf/generate.py}{\texttt{generate.py}}. Marking will be as follows:
\begin{itemize}
\item 10\%: Program compiles (if relevant) and MapReduce job(s) run to completion.
  \begin{itemize}
  \item Program doesn't compile or crashes: 0\%.
  \item Program compiles and completes successfully: 10\%.
  \end{itemize}
\item 20\%: MapReduce job(s) produce correct output on Section~\ref{sec:goal-wtf}'s data.
  \begin{itemize}
  \item Recommendations are incorrect: 0\%.
  \item Recommendations contain followed people
    or are not correctly sorted: 10\%.
  \item Output is correct: 20\%.
  \end{itemize}
\item 50\%: MapReduce job(s) produce correct output on evaluation data (undisclosed).
  \begin{itemize}
  \item Recommendations are incorrect: 0\%.
  \item Recommendations contain followed people
    or are not correctly sorted: 25\%.
  \item Output is correct: 50\%.
  \end{itemize}
\item 10\%: Program is easy to read and understand.
  \begin{itemize}
  \item It takes more than 15 minutes to understand: 0\%.
  \item It takes more than 10 minutes to understand: 5\%.
  \item It takes less than 10 minutes to understand: 10\%.
  \end{itemize}
\item 10\%: Git commit history is available and shows frequent commits.
  \begin{itemize}
  \item Commit history is not available: 0\%.
  \item Commit history is available but shows only a few large commits: 5\%.
  \item Commit history is available and demonstrates frequent commits
    showing incremental development steps: 10\%.
  \end{itemize}
\end{itemize}



\end{document}


