\documentclass[11pt]{article}

% Packages
\usepackage[colorlinks]{hyperref}
\usepackage{listings}
\usepackage{graphicx}
\usepackage[dvipsnames]{xcolor}

% Colors
\definecolor{mygray}{RGB}{235,235,235}

% Margins
\setlength{\topmargin}{-2cm}
\setlength{\textwidth}{16.5cm}
\setlength{\textheight}{24cm}
\setlength{\evensidemargin}{0cm}
\setlength{\oddsidemargin}{0cm}

% Fix link colors
\hypersetup{
    colorlinks = true,
    linkcolor=red,
    citecolor=red,
    urlcolor=blue,
    linktocpage % so that page numbers are clickable in toc
}


% Code listings
\lstset{
  basicstyle=\ttfamily,
  keywordstyle=\color{blue}\ttfamily,
  stringstyle=\color{red}\ttfamily,
  commentstyle=\color{magenta}\ttfamily,
  morecomment=[l][\color{magenta}]{\#}
}


\lstnewenvironment{cli}
                  {\footnotesize
                    \lstset{columns=fullflexible,
                      language=bash,
                      backgroundcolor=\color{mygray}
                  }}
{}

\lstnewenvironment{xml}
                  {\footnotesize
                    \lstset{columns=fullflexible,
                      language=XML,
                      backgroundcolor=\color{Salmon},
                      morekeywords={property,name,value,description,configuration}
                  }}
{}

\newcommand{\bashcode}[1]{
  \begin{footnotesize}
  \par
  \hfill \colorbox{SkyBlue}
         {
           \href{https://github.com/glatard/big-data-analytics-labs/raw/master/labs/#1}
                {(\underline{Link to file})
         }} \hfill
         \lstset{language=bash,
           columns=fullflexible,
           backgroundcolor=\color{SkyBlue}}
  \vspace*{-0.3cm}
  \lstinputlisting{#1}
  \end{footnotesize}
}

\newcommand{\javacode}[1]{
  \begin{footnotesize}
  \par
  \hfill \colorbox{YellowGreen}
         {\href{https://github.com/glatard/big-data-analytics-labs/raw/master/labs/#1}
           {(\underline{Link to file})
         }} \hfill
         \lstset{language=java,
           columns=fullflexible,
           backgroundcolor=\color{YellowGreen}}
  \vspace*{-0.3cm}
%  \lstinputlisting{#1}
  \end{footnotesize}
}

\newcommand{\textfile}[1]{
  \begin{footnotesize}
  \par
  \hfill \colorbox{SkyBlue}
         {\href{https://github.com/glatard/big-data-analytics-labs/raw/master/labs/#1}
           {(\underline{Link to file})
         }} \hfill
         \lstset{columns=fullflexible,
           backgroundcolor=\color{SkyBlue}}
  \vspace*{-0.3cm}
  \lstinputlisting{#1}
  \end{footnotesize}
}


% Notes and TODOs              
\newcommand{\postit}[1]{%
%  \noindent
%  \fcolorbox{red}{yellow}{%
%    \begin{minipage}{5cm}
%      #1
%    \end{minipage}
%   }
}

\title{\textsc{Big Data Analytics (SOEN 498/691)} \\ Laboratory sessions}

\author{Tristan Glatard\\Department of Computer Science and Software Engineering\\Concordia University, Montreal\\\href{mailto:tristan.glatard@concordia.ca}{tristan.glatard@concordia.ca}}

\begin{document}

\maketitle

\newpage

\tableofcontents

\newpage

\section{Introduction}

Now that you are able to run a simple MapReduce example using Hadoop,
we will focus on more complex algorithms. We will look at two popular
algorithms used in social networks: ``People You May Know''
(Part~\ref{part:pymk} below) and ``Who To Follow''
(Part~\ref{part:wtf}). ``People You May Know'' is not an assignment; links to 
solutions implemented in Java are provided, but it is strongly
recommended that you try to implement your own code before looking at
the solutions. ``Who To Follow'' is an assignment due for the date indicated
on the course outline.

\part{People You May Know}
\label{part:pymk}

\section{Goal}

We will implement a basic ``People You May Know''
algorithm using MapReduce. We assume an un-directed social network of
\emph{n} users with a symmetrical ``friend'' relation among users,
that is, if user A is a friend of user B then user B is also a friend
of user A. Facebook is an example of such a network. Twitter is a
counter example because user A may follow user B while user B does not
follow user A. The algorithm will recommended to user X a list of friends
$Y_i$ based on the number of friends that X and $Y_i$ have in common. The
input and output of the algorithm will be as follows:
\begin{itemize}
\item Input: A file containing \emph{n} lines with space-separated integers:
  \newline \texttt{<UID> <F$_1$> <F$_2$> ...}\newline where
  \texttt{<F$_i$>} are the friends of user
  \texttt{<UID>}. For instance: \textfile{pymk/data-small.txt}
\item Output: A file containing \emph{n} lines in the following format:
  \texttt{<UID> <R$_1$>(n$_1$) <R$_2$>(n$_2$) <R$_3$>(n$_3$)} where \texttt{<R$_i$>} are the ids of the friends recommended to user \texttt{<UID>} and $n_i$
  is the number of friends in common between \texttt{<UID>} and
  \texttt{<R$_i$>}. Recommended friends \texttt{<R$_i$>} must be ordered by
  decreasing values of \texttt{<n$_i$>}.
\end{itemize}

The following sections will guide you through a possible
implementation of such an algorithm. Feel free to follow your own
thread if you have any idea how to implement it: you may find a
better solution than the one proposed!

\section{Basic idea}

The basic idea of our initial implementation is the following:
\begin{itemize}
\item Map step:
  \begin{itemize}
  \item Receive (\_,v): \texttt{\_, <UID> <F$_1$> <F$_2$> ... <F$_k$>}
  \item Emit (k,v): \texttt{<F$_i$>},\texttt{<F$_j$>} where $i \in [1,k]$, $j \in [1,k]$ and $i \neq j$.
  \end{itemize}
  In other words, \texttt{(a,b)} \emph{and} \texttt{(b,a)} are emitted every time a friend common to users \texttt{a} and \texttt{b} is found. The following key-value pairs are emitted in response to the input example above:
  \textfile{pymk/emitted.txt}
\item Reduce step:
  \begin{itemize}
  \item Receive (k, [v]): \texttt{<UID>, [ <F$_1$> <F$_2$>
    ... ]} \newline In this list the \texttt{<F$_i$>}s are not
    unique: instead, \texttt{<F$_i$>} appears exactly \texttt{x} times
    when users \texttt{<UID>} and \texttt{<F$_i$>} have \texttt{x}
    friends in common. For instance, the following key-value pairs are received in our example:
    \textfile{pymk/received.txt}
  \item Emit (k, v): \texttt{<UID>, "<F$_1$>(n$_1$) <F$_2$>(n$_2$) ..."}\newline
    where \texttt{F$_i$} appeared exactly \texttt{$n_i$} times in the received values. In our example: \textfile{pymk/output-pymk/part-r-00000}
  \end{itemize}
\end{itemize}

Implement a ``People You May Know'' system based on this
principle. Make sure that it produces the correct result using the
example above. Then, test your implementation on larger networks by
generating input files using
\href{https://github.com/glatard/big-data-analytics-course/raw/master/labs/pymk/generate.py}{\texttt{generate.py}}.

Here is a possible solution:
\javacode{pymk/Pymk.java}

\section{Filtering existing friends}

There is a problem with the previous implementation: it may recommend
existing friends to a given user. For instance, in the example above,
user \texttt{3} is recommended to user \texttt{1} although \texttt{1}
and \texttt{3} are already friends. The information about the friends
of user \texttt{i} is currently discarded by our map step. We need to somehow
emit key-value pairs containing the friends of user \texttt{i}, to address
this issue.  Here is a possible solution:
\begin{itemize}
\item Map step:
  \begin{itemize}
  \item Receive (\_,v): \emph{no change}.
  \item Emit (k,v):
    \begin{itemize}
    \item \texttt{<F$_i$>},\texttt{<F$_j$>} where $i \in [1,k]$, $j \in [1,k]$ and $i \neq j$.
    \item \texttt{<UID>}, \color{red}\textbf{-}\color{black}\texttt{<F$_i$>} for all $i \in [1,k]$ 
    \end{itemize}
    For instance, in the example above: \textfile{pymk/emitted-1.txt}
  \end{itemize}
\item Reduce step:
  \begin{itemize}
  \item Receive (k, [v]): \texttt{<UID>, [ <F$_1$> <F$_2$>
      ... ]} \newline
    
    If \texttt{<F$_i$>} is negative then it is a friend of user \texttt{<UID>}. Otherwise, it is a user who has friends in common with \texttt{<UID>}. In our example:
    \textfile{pymk/received-1.txt}
  \item Emit (k, v): \texttt{<UID>, "<F$_1$>(n$_1$) <F$_2$>(n$_2$)
    ..."}\newline where \texttt{F$_i$} appeared exactly \texttt{$n_i$}
    times in the received values after the friends of \texttt{<UID>}
    have been removed. In our example:
    \textfile{pymk/output-pymk-1/part-r-00000}
  \end{itemize}
\end{itemize}

Implement this solution and verify that it works correctly.

Here is a possible solution:
\javacode{pymk/Pymk1.java}

\section{Jacquard similarity}

Map emits the number of friends of user for every friend. 

\newpage

\part{Who To Follow}
\label{part:wtf}

\end{document}


