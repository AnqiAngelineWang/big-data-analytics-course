\documentclass[11pt]{article}

% Packages
\usepackage[colorlinks]{hyperref}
\usepackage{listings}
\usepackage{graphicx}
\usepackage[dvipsnames]{xcolor}
\usepackage{amsmath}

% Colors
\definecolor{mygray}{RGB}{235,235,235}

% Margins
\setlength{\topmargin}{-2cm}
\setlength{\textwidth}{16.5cm}
\setlength{\textheight}{24cm}
\setlength{\evensidemargin}{0cm}
\setlength{\oddsidemargin}{0cm}

% Fix link colors
\hypersetup{
    colorlinks = true,
    linkcolor=red,
    citecolor=red,
    urlcolor=blue,
    linktocpage % so that page numbers are clickable in toc
}


% Code listings
\lstset{
  basicstyle=\ttfamily,
  keywordstyle=\color{blue}\ttfamily,
  stringstyle=\color{red}\ttfamily,
  commentstyle=\color{magenta}\ttfamily,
  morecomment=[l][\color{magenta}]{\#}
}


\lstnewenvironment{cli}
                  {\footnotesize
                    \lstset{columns=fullflexible,
                      language=bash,
                      backgroundcolor=\color{mygray}
                  }}
{}

\lstnewenvironment{xml}
                  {\footnotesize
                    \lstset{columns=fullflexible,
                      language=XML,
                      backgroundcolor=\color{Salmon},
                      morekeywords={property,name,value,description,configuration}
                  }}
{}

\newcommand{\bashcode}[1]{
  \begin{footnotesize}
  \par
  \hfill \colorbox{SkyBlue}
         {
           \href{https://github.com/glatard/big-data-analytics-labs/raw/master/labs/#1}
                {(\underline{Link to file})
         }} \hfill
         \lstset{language=bash,
           columns=fullflexible,
           backgroundcolor=\color{SkyBlue}}
  \vspace*{-0.3cm}
  \lstinputlisting{#1}
  \end{footnotesize}
}

\newcommand{\pythoncode}[1]{
  \begin{footnotesize}
  \par
  \hfill \colorbox{YellowGreen}
         {\href{https://github.com/glatard/big-data-analytics-labs/raw/master/labs/#1}
           {(\underline{Link to file})
         }} \hfill
         \lstset{language=java,
           columns=fullflexible,
           backgroundcolor=\color{YellowGreen}}
%  \vspace*{-0.3cm}
%  \lstinputlisting{#1}
  \end{footnotesize}
}

\newcommand{\textfile}[1]{
  \begin{footnotesize}
  \par
  \hfill \colorbox{SkyBlue}
         {\href{https://github.com/glatard/big-data-analytics-labs/raw/master/labs/#1}
           {(\underline{Link to file})
         }} \hfill
         \lstset{columns=fullflexible,
           backgroundcolor=\color{SkyBlue}}
  \vspace*{-0.3cm}
  \lstinputlisting{#1}
  \end{footnotesize}
}


% Notes and TODOs              
\newcommand{\postit}[1]{%
  \noindent
  \fcolorbox{red}{yellow}{%
    \begin{minipage}{5cm}
      #1
    \end{minipage}
   }
}

\title{\textsc{Big Data Analytics (SOEN 498/691)} \\ Laboratory sessions}

\author{Tristan Glatard\\Department of Computer Science and Software Engineering\\Concordia University, Montreal\\\href{mailto:tristan.glatard@concordia.ca}{tristan.glatard@concordia.ca}}

\begin{document}

\maketitle

\newpage

\tableofcontents

\newpage

\part{Apache Spark}

\section{Introduction}

Apache Spark is steadily emerging as a replacement for Hadoop
MapReduce for Big Data Analytics, for the following reasons:
\begin{itemize}
\item Spark supports in-memory computing, which is faster
  than MapReduce's file-based model.
\item Spark's programming model is much richer than
  MapReduce.
  \item Spark can run on a variety of clusters, including but not
    limited to Hadoop.
\end{itemize}

The goal of this session is to install Apache Spark on your computer
and go through simple examples to understand its main concepts. There
is no assignment associated to this session but the second lab
assignment will have to be implemented using Spark.  Most of the
material in this document is taken from the Apache Spark online
documentation:
\begin{itemize}
\item  \href{http://spark.apache.org/docs/latest/quick-start.html}{Quick Start guide}
\item \href{http://spark.apache.org/docs/latest/programming-guide.html}{Programming Guide}
\end{itemize}
Feel free to explore this documentation further!

\section{Installation}

It is assumed that you already have a working Hadoop installation on
your computer. Download Apache Spark from the
\href{http://spark.apache.org/downloads.html}{Download page}. Choose
release 2.1.0, pre-built with user-provided Hadoop. Or using the
command line:
\begin{cli}
  $ wget http://d3kbcqa49mib13.cloudfront.net/spark-2.1.0-bin-without-hadoop.tgz
\end{cli}

Unpack the release and edit \texttt{conf/spark-env.sh} so that it
contains the following, as explained
\href{http://spark.apache.org/docs/latest/hadoop-provided.html}{here}: \bashcode{spark/spark-env.sh}

Add Spark's \texttt{bin} directory to your \texttt{PATH} environment
variables so that the system can find Spark's commands:
\begin{cli}
  $ export PATH=$PATH:$PWD/spark-2.1.0-bin-without-hadoop/bin
\end{cli}

Make sure that the following example runs correctly:
\begin{cli}
  run-example SparkPi 10
\end{cli}
(\texttt{run-example} is a program located in
\texttt{\$PWD/spark-2.1.0-bin-without-hadoop/bin}. In case the output
of the previous command line is \texttt{Command not found}, check your
\texttt{PATH}.)

\section{Resilient Distributed Datasets (RDD)}

Spark programs can be written in Java, Scala, Python and R. Although
we use Python in the remainder of this session, feel free to use any
other language by following the \href{http://spark.apache.org/docs/latest/programming-guide.html}{Programming Guide}.

An easy way to test Spark is to use \texttt{pyspark}, a Spark Python shell started with:
\begin{cli}
  $ pyspark
\end{cli}
In the following, commands starting with \texttt{>>>} are typed in
\texttt{pyspark}.

Spark relies on the concept of Resilient Distributed Dataset
(RDD). RDDs are collections of elements that can be processed in
parallel. RDDs can be created from regular data structures and
files. For instance, here is how to create a RDD from an array of integers:
\begin{cli}
  >>> d = [1,2,3,4]
  >>> pd = sc.parallelize(d)
\end{cli}

Here, \texttt{sc} is the Spark Context object provided by
\texttt{pyspark}. In a standalone Spark program, you can created it as follows:
\pythoncode{spark/context.py}

In Spark, most operations are performed on RDDs. There are two types
of operations on RDDs: \emph{transformations} produce another RDD from
a given RDD, and \emph{actions} produce a simple value, e.g., an
integer, from a given RDD. \texttt{map} is an example of
transformation; the following code adds 1 to all the elements in RDD
\texttt{pd}:
\begin{cli}
  >>> inc=pd.map(lambda x: x+1)
\end{cli}
While this code looks simple, it deserves a few comments:
\begin{enumerate}
\item Spark's \texttt{map} function is \underline{very different} from
  MapReduce's map function. As we will see later, it can be used to
  implement a MapReduce map function but it is much more general than
  that.
\item In Python, a \emph{lambda} is an anonymous function. As
  explained in the
  \href{https://docs.python.org/2/reference/expressions.html#lambda}{Python
    documentation}, the expression \texttt{lambda arguments:
    expression} yields a function object. Here, it is used to pass to
  the map function a function that will be applied to all the elements
  in \texttt{pd}.
  \item At this stage, nothing has actually been computed by
    Spark. That is, \texttt{inc} only contains a reference to the
    result of the transformation which is currently not available. The
    actual value will be computed as late as possible, when it is
    really needed. This is called \emph{lazy evaluation}. 
\end{enumerate}

Now, let's use the \texttt{collect} action to return all the elements
in the RDD as a Python array:
\begin{cli}
  >>> result=inc.collect()
\end{cli}

The result of your first Spark program should appear in the shell!

\section{WordCount}

We will now implement the classical WordCount example that we had
already implemented with MapReduce. We will follow the same logic,
i.e., our program will emit a \texttt{(w,1)} pair for every word found
in the text file. Then, we will sum the \texttt{1}s associated with
a given word \texttt{w}.

First, let's create a simple two-line \texttt{test.txt} text file for testing:
\begin{cli}
  $ echo one two three two three three > test.txt
  $ echo one two three >> test.txt
\end{cli}

The first step in the Spark program is to create an RDD from
\texttt{test.txt} (this is done in \texttt{pyspark}:
\begin{cli}
  >>> lines = sc.textFile(``file://test.txt'')
\end{cli}
The RDD contains one element for every line in the text file. 

Now we need to split every line in word, which we will do using the
\texttt{flatMap} transformation.  \texttt{flatMap} is used when an
element in the input RDD is mapped to an arbitrary number of elements
(0 or more) in the output RDD. In our case, every line in the text
file will be mapped to all the words in this line:
\begin{cli}
  >>> words=lines.flatMap(lambda x: x.split())
\end{cli}
RDD \texttt{words} now contains all the words in the dataset (or it
will contain them when the program is evaluated, remember that Spark
is lazy), which we can to convert to \texttt{(w,1)} pairs using a
\texttt{map} transformation:
\begin{cli}
  >>> pairs=words.map(lambda x: (x,1))
\end{cli}
Finally, pairs need to be ``reduced'', i.e., they need to be grouped
by their first elements and the corresponding \texttt{1}s must be
summed up. This is done using the \texttt{reduceByKey} transformation:
\begin{cli}
  >>> counts=pairs.reduceByKey(lambda x,y: x+y)
\end{cli}
The function passed to \texttt{reduceByKey} is a binary
operator that takes as argument two values from the key-value pairs
and returns one. Besides, this function must be
\underline{commutative} and \underline{associative}. 

The word counts can now be printed using the \texttt{foreach} transformation:
\begin{cli}
  >>> def g(x):
  ...   print x
  >>> counts.foreach(lambda x: g(str(x[0])+'': ''+str(x[1]))
\end{cli}
Or they might be saved in a text file using the \texttt{saveAsTextFile} action:
\begin{cli}
  >>> counts.saveAsTextFile(``file:///tmp/counts'')
\end{cli}

Here is a complete version of WordCount in Spark:
\pythoncode{spark/wordcount.py}

To run this program, you will have to update your environment as follows:
\begin{cli}
  $ export PYTHONPATH=$PWD/spark-2.1.0-bin-without-hadoop/python
  $ sudo pip install py4j
\end{cli}

The program is run as follows:
\begin{cli}
  $ ./wordcount.py file:///tmp/test.txt file:///tmp/counts
\end{cli}

Spark can transparently work with files stored on HDFS. Start your
HDFS daemon, upload \texttt{test.txt} to HDFS and re-run the WordCount program:
\begin{cli}
  $ ./wordcount.py hdfs://test.txt file:///tmp/counts-hdfs
\end{cli}

\section{Going further}

A complete list of transformations and actions is available in the
\href{http://spark.apache.org/docs/latest/programming-guide.html}{Programming
  Guide}. Examples are also available in Spark's
\href{https://github.com/apache/spark}{Github repository}.

Try re-implementing in Spark the kmeans clustering algorithm
programmed in a
\href{https://github.com/glatard/big-data-analytics-course/releases/download/0.7.3/kmeans.pdf}{previous
  lab session} using MapReduce. A solution is available
\href{https://github.com/apache/spark/tree/master/examples/src/main/python}{there}.

\end{document}


