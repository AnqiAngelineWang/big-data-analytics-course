\documentclass[11pt]{article}

% Packages
\usepackage[colorlinks]{hyperref}
\usepackage{listings}
\usepackage{graphicx}
\usepackage[dvipsnames]{xcolor}
\usepackage{amsmath}

% Colors
\definecolor{mygray}{RGB}{235,235,235}

% Margins
\setlength{\topmargin}{-2cm}
\setlength{\textwidth}{16.5cm}
\setlength{\textheight}{24cm}
\setlength{\evensidemargin}{0cm}
\setlength{\oddsidemargin}{0cm}

% Fix link colors
\hypersetup{
    colorlinks = true,
    linkcolor=red,
    citecolor=red,
    urlcolor=blue,
    linktocpage % so that page numbers are clickable in toc
}


% Code listings
\lstset{
  basicstyle=\ttfamily,
  keywordstyle=\color{blue}\ttfamily,
  stringstyle=\color{red}\ttfamily,
  commentstyle=\color{magenta}\ttfamily,
  morecomment=[l][\color{magenta}]{\#}
}


\lstnewenvironment{cli}
                  {\footnotesize
                    \lstset{columns=fullflexible,
                      language=bash,
                      backgroundcolor=\color{mygray}
                  }}
{}

\lstnewenvironment{xml}
                  {\footnotesize
                    \lstset{columns=fullflexible,
                      language=XML,
                      backgroundcolor=\color{Salmon},
                      morekeywords={property,name,value,description,configuration}
                  }}
{}

\newcommand{\bashcode}[1]{
  \begin{footnotesize}
  \par
  \hfill \colorbox{SkyBlue}
         {
           \href{https://github.com/glatard/big-data-analytics-labs/raw/master/labs/#1}
                {(\underline{Link to file})
         }} \hfill
         \lstset{language=bash,
           columns=fullflexible,
           backgroundcolor=\color{SkyBlue}}
  \vspace*{-0.3cm}
  \lstinputlisting{#1}
  \end{footnotesize}
}

\newcommand{\pythoncode}[1]{
  \begin{footnotesize}
  \par
  \hfill \colorbox{YellowGreen}
         {\href{https://github.com/glatard/big-data-analytics-labs/raw/master/labs/#1}
           {(\underline{Link to file})
         }} \hfill
         \lstset{language=java,
           columns=fullflexible,
           backgroundcolor=\color{YellowGreen}}
%  \vspace*{-0.3cm}
%  \lstinputlisting{#1}
  \end{footnotesize}
}

\newcommand{\textfile}[1]{
  \begin{footnotesize}
  \par
  \hfill \colorbox{SkyBlue}
         {\href{https://github.com/glatard/big-data-analytics-labs/raw/master/labs/#1}
           {(\underline{Link to file})
         }} \hfill
         \lstset{columns=fullflexible,
           backgroundcolor=\color{SkyBlue}}
  \vspace*{-0.3cm}
  \lstinputlisting{#1}
  \end{footnotesize}
}

\newcounter{ques}
\setcounter{ques}{1}
\newcommand{\question}[1]{\paragraph{}\noindent\textbf{Q\theques} - #1\stepcounter{ques} }
\newcommand{\answer}[0]{
\vspace*{0.5cm}
  \noindent Expected output on \texttt{iliad} and \texttt{odyssey}:}

% Notes and TODOs              
\newcommand{\postit}[1]{%
  \noindent
  \fcolorbox{red}{yellow}{%
    \begin{minipage}{5cm}
      #1
    \end{minipage}
   }
}

\title{\textsc{Big Data Analytics (SOEN 498/691)} \\ Laboratory sessions}

\author{Tristan Glatard\\Department of Computer Science and Software Engineering\\Concordia University, Montreal\\\href{mailto:tristan.glatard@concordia.ca}{tristan.glatard@concordia.ca}}

\begin{document}

\maketitle

\newpage

\tableofcontents

\newpage

\part{Log analysis}

\section{Introduction}

In this assignment you will analyze log files using Apache
Spark. We will focus on Linux syslog files and more precisely on the
\texttt{messages} file located in \texttt{/var/log} on most Linux
distributions. Such analyses have various applications in system
administration, e.g., for security diagnostics.

Lines in \texttt{/var/log/messages} are formatted as follows:
\begin{center}
  \texttt{<timestamp> <host\_name> <source>: <message>}
\end{center}
where \texttt{<timestamp>} is a timestamp, \texttt{<host\_name>} is
the host (computer) name, \texttt{<source>} is the program that
generated the line and \texttt{<message>} is the log message. Here is
an example:
\begin{center}
  \texttt{Feb 28 03:30:01 iliad systemd: Started Session 4415 of user achille.}
\end{center}

You have to write a Spark program to answer the \textbf{9 questions}
detailed below (\textbf{Q1}-\textbf{Q9}). Log files collected on two
hosts called \texttt{iliad} and \texttt{odyssey} are available on
Moodle to test your program. During marking, your program will be run
on similar but different files to check its correctness.
Section~\ref{sec:submission} summarizes submission instructions and
the marking scheme that will be used to evaluate your submission. You
are strongly encouraged to review these instructions before starting
the assignment.

\section{Filtering and Counting}

The following queries must be run separately on the log files of each
host, i.e., \emph{one answer must be returned for each host}.

\question{For each host, print the total number of available log
  lines.}

\answer
\begin{cli}
$  ./log_analyzer -q 1 iliad odyssey
* Q1: line counts
  + iliad: 63854                                                                
  + odyssey: 65405
\end{cli}

\question{For each host, print the number of sessions that were
  started for user \texttt{achille}. Session starts are logged in
  messages containing the following string: \texttt{Starting Session
    <id> of user <user>}, where \texttt{<id>} is the session id and
  \texttt{<user>} is the user name.}

\answer
\begin{cli}
$  ./log_analyzer -q 2 iliad odyssey
* Q2: sessions of user 'achille'
  + iliad: 5173
  + odyssey: 5228
\end{cli}

\question{For each host, list the unique user names who started a session.}

\answer
\begin{cli}
$  ./log_analyzer -q 3 iliad odyssey
* Q3: unique user names
  + iliad: ['gaia', 'pollux', 'achille', 'helene', 'hector']                    
  + odyssey: ['achille', 'hector', 'ares']  
\end{cli}

\question{For each host, list the number of sessions started per user.}

\answer
\begin{cli}
$  ./log_analyzer -q 4 iliad odyssey
* Q4: sessions per user
  + iliad: [('gaia', 2), ('pollux', 38), ('achille', 5173), ('helene', 248), ('hector', 9)]
  + odyssey: [('achille', 5228), ('hector', 2), ('ares', 40)]  
\end{cli}

\section{Errors}

The following queries must be run separately on the logs of each host,
i.e., \emph{one answer must be returned for each host}.

\question{For each host, count the error messages, i.e., the lines that contain string ``error'' (case insensitive match).}

\answer
\begin{cli}
$  ./log_analyzer -q 5 iliad odyssey
* Q5: number of errors
  + iliad: 2723
  + odyssey: 25805
\end{cli}

\question{For each host, print the 5 most frequent error messages and their counts.}

\answer
\begin{cli}
$  ./log_analyzer -q 6 iliad odyssey
* Q6: 5 most frequent error messages
  + iliad: 
    - (889, 'journal: ethtool ioctl error: No such device ')
    - (24, 'gnome-session: ** (evince:31187): WARNING **: Error setting file metadata: No such file or directory ')
    - (9, 'gnome-session: https://yum.dockerproject.org/repo/main/centos/7/repodata/3849c07e5505140b8134d3cf1bef35cd1cfe4797bee74799953d73a568694fd1-filelists.sqlite.bz2: [Errno 14] HTTPS Error 403 - Forbidden ')
    - (9, "gnome-session: GDBus.Error:org.gtk.GDBus.UnmappedGError.Quark._imsettings_2derror_2dquark.Code5: Current desktop isn't targeted by IMSettings. ")
    - (8, 'firefox.desktop: Crash Annotation GraphicsCriticalError: |[0][GFX1-]: GLContext is disabled due to a previous crash.|[6][GFX1-]: GLContext is disabled due to a previous crash.|[7][GFX1-]: GLContext is disabled due to a previous crash.|[8][GFX1-]: GLContext is disabled due to a previous crash.|[4][GFX1-]: GLContext is disabled due to a previous crash.|[5][GFX1-]: GLContext is disabled due to a previous crash.[GFX1-]: GLContext is disabled due to a previous crash. ')
  + odyssey:                                                                    
    - (9229, 'gnome-session: (tracker-miner-fs:30474): Tracker-CRITICAL **: Could not execute sparql: column nie:url is not unique (strerror of errno (not necessarily related): No such file or directory) ')
    - (4519, 'gnome-session: (tracker-miner-fs:30474): Tracker-CRITICAL **: (Sparql buffer) Error in task 0 of the array-update: column nie:url is not unique (strerror of errno (not necessarily related): No such file or directory) ')
    - (2776, 'gnome-session: (tracker-miner-fs:30474): Tracker-CRITICAL **: (Sparql buffer) Error in task 2 of the array-update: column nie:url is not unique (strerror of errno (not necessarily related): No such file or directory) ')
    - (2401, 'gnome-session: (tracker-miner-fs:1259): Tracker-CRITICAL **: Could not execute sparql: column nie:url is not unique (strerror of errno (not necessarily related): No such file or directory) ')
    - (1697, 'gnome-session: (tracker-miner-fs:1259): Tracker-CRITICAL **: (Sparql buffer) Error in task 0 of the array-update: column nie:url is not unique (strerror of errno (not necessarily related): No such file or directory) ')
\end{cli}

\section{Combining logs}

The following queries are run on a combination of both hosts, i.e.,
\emph{a single answer combining information coming from both hosts
  must be returned for each question}.

\question{List the user names who started a session on both hosts.}

\answer
\begin{cli}
$  ./log_analyzer -q 7 iliad odyssey
* Q7: users who started a session on both hosts, i.e., on exactly 2 hosts.
  + : ['achille', 'hector']      
\end{cli}

\question{List the user names who started a session on exactly one host and list this host.}

\answer
\begin{cli}
$  ./log_analyzer -q 8 iliad odyssey
* Q8: users who started a session on exactly one host, with host name.
  + : [('gaia', 'iliad'), ('pollux', 'iliad'), ('helene', 'iliad'), ('ares', 'odyssey')]
\end{cli}

\section{Anonymization}

\question{Anonymize the logs, i.e., replace user names with strings
  formatted as \texttt{user-<i>} where \texttt{i} is the index of the
  user name in the array of original user names sorted
  alphabetically. For instance, user names \texttt{foo} and
  \texttt{bar} must be replaced by \texttt{user-1} and \texttt{user-0}
  (respectively), assuming that \texttt{foo} and \texttt{bar} are the
  only user names in the log. In addition to writing the anonymized
  files, your program must print the mapping used for the
  anonymization and the location where the anonymized files were written.}

\answer
\begin{cli}
$  ./log_analyzer -q 9 iliad odyssey
  + iliad:                                                                      
  .    User name mapping: [('achille', 'user-0'), ('gaia', 'user-1'), ('hector', 'user-2'),\
    ('helene', 'user-3'), ('pollux', 'user-4')]
.    Anonymized files: iliad-anonymized-10
  + odyssey:                                                                    
.    User name mapping: [('achille', 'user-0'), ('ares', 'user-1'), ('hector', 'user-2')]
.    Anonymized files: odyssey-anonymized-10
\end{cli}
The corresponding anonymized logs are available on Moodle.

\newpage

\section{Submission and Marking}
\label{sec:submission}

Your assignment must be submitted as a \texttt{tgz} or \texttt{zip}
archive on Moodle. The archive name must contain your name(s) and
student id(s). If you are submitting in a team of two, each team
member must submit a copy of the assignment. The submitted archive has
to include the following files:
\begin{itemize}
\item A command-line program that prints the output of the above
  questions to the console, using the following syntax:
\begin{cli}
$ ./log_analyzer -q <i> <dir1> <dir2> 
\end{cli}
Where \texttt{<dir1>} and \texttt{<dir2>} are two directories
containing log files collected on two different hosts, and
\texttt{<i>} is the question number. For instance, your program will
be executed as follows to run question 7 on the test logs:
\begin{cli}
  $ log_analyzer -q 7 iliad odyssey
\end{cli}
\item The source code of your program. You may use any programming
  language supported by Spark, i.e., Java, Scala or Python.
\item A README file explaining how to compile (if relevant) your
  program, and any other useful information to run it, e.g.,
  required versions, required environment variables, etc.
\item A Git commit history (\texttt{.git} directory) or the URL of a
  Git repository (e.g., Github) that you used to develop your program. Your
  commit history will be checked and frequent commits showing
  incremental development will be favored over monolithic ones.
\end{itemize}

Your program will be run on the testing data (\texttt{iliad} and
\texttt{odyssey}) and on undisclosed
\texttt{/var/log/messages} files collected on Linux hosts. The following
marking scheme will be used:
\begin{itemize}
\item 5 points: program compiles (if relevant) and runs with the imposed syntax. 
\item For each question (9 questions in total):
  \begin{itemize}
  \item 2 points: program runs to completion.
    \begin{itemize}
    \item Program runs to completion out of the box: 2 points.
    \item Program runs to completion after minor fixes: 1 point.
    \item Program runs to completion after major fixes or cannot be fixed: 0 point.
    \end{itemize}
  \item 2 points: program gives correct answer on testing data (\texttt{iliad} and \texttt{odyssey}).
    \begin{itemize}
    \item Program gives totally correct answer: 2 points.
    \item Program gives partially correct answer (e.g. incomplete lists): 1 point.
    \item Program gives incorrect answer: 0 point.
    \end{itemize}
  \item 4 points: program gives correct answer on undisclosed evaluation data.
    \begin{itemize}
    \item Program gives totally correct answer: 4 points.
    \item Program gives partially correct answer: 2 points.
    \item Program gives incorrect answer: 0 point.
    \end{itemize}
  \item 2 points: program uses mainly Spark operations (transformation and actions).
    \begin{itemize}
    \item Program uses mainly Spark operations: 2 points.
    \item Program uses some Spark operations: 1 point. 
    \item Program does not use Spark operations: 0 point.
    \end{itemize}
  \end{itemize}
\item 5 points: Git commit history is available and shows frequent commits.
  \begin{itemize}
    \item Commit history is available and demonstrates frequent commits showing incremental
      development steps: 5 points.
    \item Commit history is available but shows only a few large commits: 2 points.
    \item Commit history is not available: 0 point. 
  \end{itemize}
\end{itemize}
Total: 100 points (5+9$\times$10+5).

\end{document}


